\subsection{Partie 1 : Les objets}
Commençons par créer les différentes classes
que nous allons manipuler.\\ Vous allez devoir créer plusieurs classes:
\begin{itemize}
  \item CPU
  \item ALU
  \item Decoder
  \item Environment
  \item Instruction
  \item ProgramLoader 
\end{itemize} 
NB: Relisez le cours afin de
        savoir quelle classe possède quel champ. Il peut être intéressant que le
        CPU connaisse son environnement et inversement.

\subsection{Partie 2 : Les Instruction} La classe Instruction prendra dans le
constructeur la valeur de tous les éléments (obtenus via le Decoder que l'on
verra plus tard) de tous les types d'instruction et les rendra accessibles via
des champs.\\ Par exemple: 
\begin{code} 
  Instruction i = new Instruction(rd, rs, rt, ......); 
  int a = i.Rd;
  int b = i.Rs;
  int c = i.Immediate;
  int d = i.TargetAddress;
\end{code}
Comme vous pouvez le voir, on ne s'occupe pas pour
l'instant du type d'instruction, on utilise les champs des 3 instructions et en
fonction de son type que l'on verra plus tard, on ne se servira que de certains
champs.

\subsection{Partie 3 : Le ProgramLoader}
ProgramLoader sera une classe static
qui contiendra une methode du nom de votre choix qui prendra en paramètre un
environment, un path et qui va charger en RAM le binaire.\\

En utilisant les fonctions de lecture de fichiers vous aller devoir charger le
binaire dans la RAM de votre programme. Il suffit juste de lire byte par byte et
de mettre la donnée à la bonne adresse mémoire.

\subsection{Partie 4 : Le décodage des instructions}

Grace aux fonctions de bit masking ainsi que le cours que vous venez de lire,
vous allez devoir décoder une instruction.  Ce n'est pas dans cette partie que
vous aurez à gérer le type de l'instruction.\\

Voici quelques exemples qui pourons vous servir :
\begin{code} 
  int a = 0x0000FABB & 0x0000000F;
  // a = 0x0000000B
  int b = 0x0000000C | 0x00000003;
  // b = 0x0000000F
  int c = (0x0030FABB & 0x00F00000) >> 24;
  // c = 0x00000003
\end{code}


\subsection{Partie 5 : La stack}
Dans la classe CPU, nous allons faire les
fonctions de manipulation de la stack. Votre implémentation devra contenir des
fonction push et pop qui prennent un tableau de byte en paramètre et qui empile
tous les élements de ce tableau (attention: l'élement 0 du tableau devra être au
sommet).  Une fois cette fonction faite, créez d'autre surcharges qui prennent
en paramètre soit un int (32 bits) soit un short (16 bits) soit un byte(8 bits).
\\ Vous utiliserez ensuite la classe BitConverter qui va vous permettre de
transformer un type primitif en bytes et vice-versa.
\begin{code}
  int a = 5;
  byte[] bytes_arr = BitConverter.GetBytes(a);
  int b =
  BitConverter.ToInt32(bytes_arr);
\end{code}
Vous pouvez aussi utiliser ToInt16(), servez-vous de ces fonctions pour 
implémenter de la même manière les fonctions pop.
\begin{code}
  #region stack push
        public void stack_push(int data);

        public void stack_push(short data);

        public void stack_push(byte data);

        public void stack_push(byte[] data); #endregion

  #region stack pop
        public int stack_pop_int();
        
        public short stack_pop_short();
        
        public byte stack_pop_byte();
        
        public byte[] stack_pop_bytes(int size);
  #endregion
\end{code}

Une fois les fonctions stack\_push(byte[] data) et stack\_pop\_bytes(int size)
créées, les autres fonctions se font en une seule ligne.\\ Si vous vous demandez
l'intérêt de ceci, c'est parce que vous aurez des instructions qui pourront push
ou pop soit 32 bits, soit 16 bits soit 8 bits.

Le CPU possède un registre stack pointer (SP) qui pointe vers le sommet de la
pile, il s'agit du 30ième registre général (index 29).\\ Pensez à modifier votre
SP lors des fonctions push et pop afin de le mettre à jours, servez-vous en
également pour vérifier qu'une opération ne déplace pas le SP au delà des
limites de la stack. Un getter et setter pour un champ SP de la classe CPU
seraient intéressants. Un champ StackBase également afin de ne pas pop sur une
pile vide. N'hésitez pas à déclencher une exception.
\begin{code}
  throw new Exception("Votre message");
\end{code}

\subsection{Partie 6 : L'ALU}

L'ALU contiendra toutes les fonctions arithmétiques et logiques. L'idéal serait
une fonction qui prenne un objet Instruction en paramètre et appelle la fonction
correspondante.  La classe Dictionnary<type1, type2> vous permet d'établir une
correspondance entre des clefs de type \textit{type1} et des valeurs de type
\textit{type2}.\\ Utilisez un dictionnaire permettant d'associer un opcode/funct
à un delegate vous permettra de récuperer facilement la fonction correspondante
à une instruction.\\ Voici quelques exemples d'utilisation de la classe
dictionnaire qui vous seront utiles:
\begin{code}
  // On veut associer un identifiant (clef) à une chaine de caractères (valeur)
  Dictionary<int, string> ids = new Dictionary<int, string>();

  ids.Add(3, "Hello!"); // La clef 3 est associée à la valeur "Hello!"
  ids.Add(1, "test");
  ids.Add(42, "the game");

  // Retourne la valeur associée à 3
  string value = ids[3];
  // Une exception est levée si la clef n'existe pas
  string error = ids[8];

  /*
  La methode ContainsKey vous permettra de savoir si une clef est présente ou
  non
  */
  ids.ContainsKey(1); // Renvoie vrai
  ids.ContainsKey(13); // Renvoie faux
\end{code}
